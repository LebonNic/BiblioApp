%------------------------------------------------
\PassOptionsToPackage{utf8}{inputenc}
\usepackage{inputenc}
%------------------------------------------------
\newcommand{\titreRapport}{Rapport d'intégration d'application}
\newcommand{\sousTitreRapport}{Valorisation d'une base de données avec Google App Engine et client web}
\newcommand{\filiere}{Filière Génie Logiciel et Systèmes Informatiques}
\newcommand{\noms}{Antoine \textsc{Colmard} et Nicolas \textsc{Prugne}}
\newcommand{\ecole}{\small{
	\textsc{Institut Supérieur d'Informatique \\de Modélisation et de leurs Applications}\\[0.5em]
	Campus des Cézeaux \\
	24 Av. des Landais \\
	BP 10125 \\
	63173 \textsc{Aubiere} CEDEX}}
\newcommand{\dateRapport}{Décembre 2014}

%----------------------------------------------------------------------------------------
% Hyper References
%----------------------------------------------------------------------------------------
\usepackage[pdftex, hyperfootnotes=false, pdfpagelabels]{hyperref}
\pdfcompresslevel=9
\pdfadjustspacing=1

\hypersetup{
colorlinks=true, linktocpage=true, pdfstartpage=1, pdfstartview=FitV,
breaklinks=true, pdfpagemode=UseNone, pageanchor=true, pdfpagemode=UseOutlines,
plainpages=false, bookmarksnumbered, bookmarksopen=true, bookmarksopenlevel=1,
hypertexnames=true, pdfhighlight=/O, urlcolor=black, linkcolor=black, citecolor=black,
%------------------------------------------------
% PDF file meta-information
%------------------------------------------------
pdftitle={\sousTitreRapport},
pdfauthor={\noms},
pdfsubject={},
pdfkeywords={},
pdfcreator={pdfLaTeX},
pdfproducer={LaTeX with hyperref and classicthesis}
%------------------------------------------------
}

%----------------------------------------------------------------------------------------
% Packages
%----------------------------------------------------------------------------------------
%--- Babel --------------------------------------
\usepackage[T1]{fontenc}
\usepackage[francais,american]{babel}
%--- Classic Thesis -----------------------------
\usepackage[
	eulerchapternumbers,
	manychapters,
	listings,
	pdfspacing,
	subfig,
	beramono,
	eulermath,
	parts,
	dottedtoc
	]{classicthesis}
\AtBeginDocument{\renewcommand{\thepart}{\Roman{part}}}
%--- Maths --------------------------------------
\usepackage[square, numbers]{natbib}
\usepackage[fleqn]{amsmath}
%--- Spacing and geometry -----------------------
\usepackage{xspace}
\usepackage{setspace}
\usepackage[pdftex]{graphicx}
\usepackage[total={16cm,24.7cm},centering,includefoot]{geometry}
\linespread{1.50}
%--- Patches for latex --------------------------
\usepackage{mparhack}
\usepackage{fixltx2e}
%--- Glossaries ---------------------------------
\usepackage[acronym,style=long,nolist]{glossaries}
\makeglossaries
%--- Table of content configuration -------------
\usepackage[nottoc,notlof,numbib]{tocbibind}
%------------------------------------------------
\usepackage{lipsum}
\usepackage{indentfirst}

\usepackage[official]{eurosym}
%----------------------------------------------------------------------------------------
% Commands
%----------------------------------------------------------------------------------------
\graphicspath{{img/}}
\newcommand{\fig}[4][1]{\begin{figure}[!ht]
	\centering
	\includegraphics[scale=#1]{#3}
	\caption{#4}
	\label{#2}
\end{figure}}

\newcounter{dummy}
%----------------------------------------------------------------------------------------
% Algorithms
%----------------------------------------------------------------------------------------
\usepackage[section]{algorithm}
\usepackage{algpseudocode}
% Francisation des algorithmes
\floatname{algorithm}{Algorithme}
\renewcommand{\algorithmicfunction}{\textbf{fonction}}
\renewcommand{\algorithmicwhile}   {\textbf{tant que}}
\renewcommand{\algorithmicdo}      {\textbf{faire}}
\renewcommand{\algorithmicend}     {\textbf{fin}}
\renewcommand{\algorithmicif}      {\textbf{si}}
\renewcommand{\algorithmicelse}    {\textbf{sinon}}
\renewcommand{\algorithmicthen}    {\textbf{alors}}
\renewcommand{\algorithmicfor}     {\textbf{pour}}
\renewcommand{\algorithmicforall}  {\textbf{pour tout}}
\renewcommand{\algorithmicdo}      {\textbf{faire}}
\renewcommand{\algorithmicrepeat}  {\textbf{répéter}}
\renewcommand{\algorithmicuntil}   {\textbf{jusqu'à}}
\let\mylistof\listof
\renewcommand\listof[2]{\mylistof{algorithm}{Table des algorithmes}}
\makeatletter
\renewcommand{\thealgorithm}{\arabic{algorithm}} 
\makeatother

%----------------------------------------------------------------------------------------
%	Tables, figures
%----------------------------------------------------------------------------------------
\usepackage{caption}
\captionsetup{format=hang,font=small}
\usepackage{subfig}
\addto\captionsfrench{\renewcommand\listfigurename{Table des figures}}
\setcounter{table}{-1}
\addto\captionsfrench{\renewcommand\listtablename{Table des tableaux}}
\addto\captionsfrench{\renewcommand\tablename{\textsc{Tableau}}}
%----------------------------------------------------------------------------------------
% Bibliographie
%----------------------------------------------------------------------------------------
\bibliographystyle{alpha}
\addto\captionsfrench{\renewcommand{\bibname}{Références bibliographiques}}
%----------------------------------------------------------------------------------------
% Code listing
%----------------------------------------------------------------------------------------
\usepackage{listings} 
\lstset{language=C++, % Specify the language for listings here
keywordstyle=\color{RoyalBlue}, % Add \bfseries for bold
basicstyle=\small\ttfamily, % Makes listings a smaller font size and a different font
%identifierstyle=\color{NavyBlue}, % Color of text inside brackets
commentstyle=\color{Green}\ttfamily, % Color of comments
stringstyle=\rmfamily, % Font type to use for strings
numbers=left, % Change left to none to remove line numbers
numberstyle=\scriptsize, % Font size of the line numbers
stepnumber=5, % Increment of line numbers
numbersep=8pt, % Distance of line numbers from code listing
showstringspaces=false, % Sets whether spaces in strings should appear underlined
breaklines=true, % Force the code to stay in the confines of the listing box
%frameround=ftff, % Uncomment for rounded frame
frame=lines, % Frame border - none/leftline/topline/bottomline/lines/single/shadowbox/L
backgroundcolor=\color{black!5}, % set backgroundcolor
belowcaptionskip=.75\baselineskip, % Space after the "Listing #: Desciption" text and the listing box
inputencoding=utf8,
extendedchars=true,
literate={à}{{\`a}}1 {â}{{\^a}}1 {é}{{\'e}}1 {è}{{\`e}}1 {ê}{{\^e}}1 {°}{{\degre}}1 {ô}{{\^o}}1 {î}{{\^i}}1 {ç}{{\c c}}1
}

%----------------------------------------------------------------------------------------
%	BACKREFERENCES
%----------------------------------------------------------------------------------------
\usepackage{ifthen} % Allows the user of the \ifthenelse command
\newboolean{enable-backrefs} % Variable to enable backrefs in the bibliography
\setboolean{enable-backrefs}{false} % Variable value: true or false

\newcommand{\backrefnotcitedstring}{\relax} % (Not cited.)
\newcommand{\backrefcitedsinglestring}[1]{(Cited on page~#1.)}
\newcommand{\backrefcitedmultistring}[1]{(Cited on pages~#1.)}
\ifthenelse{\boolean{enable-backrefs}} % If backrefs were enabled
{
\PassOptionsToPackage{hyperpageref}{backref}
\usepackage{backref} % to be loaded after hyperref package 
\renewcommand{\backreftwosep}{ and~} % separate 2 pages
\renewcommand{\backreflastsep}{, and~} % separate last of longer list
\renewcommand*{\backref}[1]{}  % disable standard
\renewcommand*{\backrefalt}[4]{% detailed backref
\ifcase #1 
\backrefnotcitedstring
\or
\backrefcitedsinglestring{#2}
\else
\backrefcitedmultistring{#2}
\fi}
}{\relax}

%\usepackage{lmodern}
